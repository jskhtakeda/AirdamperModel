\newcommand{\paren}[1]{\left({#1}\right)}
\newcommand{\abs}[1]{\left|{#1}\right|}

\documentclass{jsarticle}

\usepackage{amsmath,amssymb}
\usepackage{mathtools}
\usepackage{cases}

\begin{document}

断熱変化を仮定

\begin{align}
  \frac{p}{p_0} = \paren{\frac{\rho}{\rho_0}}^{\gamma}
\end{align}

質量保存則

\begin{align}
  \rho_0 A_p h_p = \rho A_p \paren{h_p - y} + \int_0^t Q dt
\end{align}

ベルヌーイ流れを仮定

\begin{align}
  Q = \it{sgn} \paren{p - p_0} c_0 A_0 \sqrt{2 \rho_0 \abs{p - p_0}}
\end{align}

$\rho$と$Q$を代入消去して

\begin{align}
  1 = \paren{\frac{p}{p_0}}^{\frac{1}{\gamma}} \paren{1 - \frac{y}{h_p}} + \frac{c_0 A_0 \sqrt{2 \rho_0}}{\rho_0 A_p h_p} \int_0^t \it{sgn} \paren{p - p_0} \sqrt{p_0} \sqrt{\abs{\frac{p}{p_0} - 1}} dt
\end{align}

$P = \frac{p - p_0}{p_0}$,$Y = \frac{y}{h_p}$とおいて

\begin{align}
  1 = \paren{1 + P}^{\frac{1}{\gamma}} \paren{1 - Y} + \frac{c_0 A_0}{A_p h_p} \sqrt{\frac{2 p_0}{\rho_0}} \it{sgn} \paren{P} \int_0^t \sqrt{\abs{P}} dt
\end{align}

$C = \frac{c_0 A_0}{A_p h_p} \sqrt{\frac{2 p_0}{\rho_0}}$とおいて

\begin{align}
  \paren{1 - Y} \paren{1 + P}^{\frac{1}{\gamma}} = 1 - C \it{sgn} \paren{P} \int_0^t \sqrt{\abs{P}} dt
\end{align}

$t$で微分

\begin{align}
  \frac{1}{\gamma} \paren{1 - Y} \paren{1 + P}^{\frac{1}{\gamma} - 1} \frac{dP}{dt} + C \it{sgn} \paren{P} \sqrt{\abs{P}} = (1 + P)^\frac{1}{\gamma} \frac{dY}{dt}
\end{align}

ここで,運動方程式は

\begin{align}
  m \ddot{y} &= m g - \paren{p - p_0} A_p \\
  m h_p \ddot{Y} &= m g - P p_0 A_p \\
  P &= \frac{mg}{p_0 A_p} \paren{1 - \frac{h_p \ddot{Y}}{g}} \\
\end{align}

$\frac{m g}{p_0 A_p} = D$,$Y = \frac{y}{h_p}$にして

\begin{align}
  P = D \paren{1 - \frac{\ddot{y}}{g}}
\end{align}

なのでこれを代入して$P$を消去

\begin{align}
  \frac{1}{\gamma}
  \paren{1 - \frac{y}{h_p}}
  \paren{D \paren{1 - \frac{\ddot{y}}{g}} + 1}^{\frac{1}{\gamma} - 1}
  \paren{- D \frac{\dddot{y}}{g}}
  + C \it{sgn} \paren{1 - \frac{\ddot{y}}{g}} \sqrt{D \abs{1 - \frac{\ddot{y}}{g}}}
  = \paren{D \paren{1 - \frac{\ddot{y}}{g}} + 1}^\frac{1}{\gamma} \frac{\dot{y}}{h_p}
\end{align}

$x = y$,$v = \dot{y}$,$a = \dot{v} = \ddot{y}$とおくと,

\begin{numcases}
  {}
  \dot{x} = v & \\
  \dot{v} = a & \\
  \dot{a} = \frac{
    C \it{sgn} \paren{1 - \frac{a}{g}} \sqrt{D \abs{1 - \frac{a}{g}}}
    - \paren{D \paren{1 - \frac{a}{g}} + 1}^{\frac{1}{\gamma}} \frac{v}{h_p}
  }{
    \frac{1}{\gamma} \paren{1 - \frac{x}{h_p}}
    \paren{D \paren{1 - \frac{a}{g}} + 1}^{\frac{1}{\gamma} - 1} \frac{D}{g}
    } &
\end{numcases}

初期値$t = 0$(衝突直前)で$x = 0$,$v = v$,$a = g$として,4次のルンゲクッタ

\end{document}
